\documentclass[25pt]{sciposter}

\usepackage[T1]{fontenc}
\usepackage[utf8]{inputenc}

\usepackage{amsthm}

\usepackage[dvipsnames,usenames,svgnames,table]{xcolor} 
\usepackage{lipsum}
\usepackage{epsfig}
\usepackage{amsmath}
\usepackage{amssymb}

\usepackage[english]{babel}
\usepackage{geometry}
\usepackage{multicol}
\usepackage{graphicx}
\usepackage{tikz}
\usepackage{wrapfig}
\usepackage{gensymb}
\usepackage{tocloft}
\usepackage{empheq}

\usepackage{pgfplots}
\pgfplotsset{width=11cm,compat=1.9}


% for nice tableas
\usepackage{booktabs}

\graphicspath{ {img/} }

\geometry{
	landscape,
	a1paper,
	left=5mm,
	right=50mm,
	top=5mm,
	bottom=50mm,
}
\usepackage{array}   % for \newcolumntype macro
\newcolumntype{L}{>{$}m{5.5cm}<{$}} % math-mode version of "l" column type

%BEGIN LISTINGDEF


\newtheorem*{lem}{Lemma}


\newcommand*\widefbox[1]{\fbox{\hspace{2em}#1\hspace{2em}}}
\newcommand{\limm}{\lim\limits_{n \to \infty}}
\newcommand{\limx}[1]{\lim\limits_{x \to #1}}
\newlength\dlf  % Define a new measure, dlf
\newcommand\alignedbox[2]{
	% Argument #1 = before & if there were no box (lhs)
	% Argument #2 = after & if there were no box (rhs)
	&  % Alignment sign of the line
	{
		\settowidth\dlf{$\displaystyle #1$}
		% The width of \dlf is the width of the lhs, with a displaystyle font
		\addtolength\dlf{\fboxsep+\fboxrule}
		% Add to it the distance to the box, and the width of the line of the box
		\hspace{-\dlf}
		% Move everything dlf units to the left, so that & #1 #2 is aligned under #1 & #2
		\boxed{#1 #2}
		% Put a box around lhs and rhs
	}
}
\usepackage{graphicx,url}

%BEGIN TITLE
\title{\huge{Information Security}}

\author{\large{David Zollikofer}}
%END TITLE


%\usepackage{eulervm}
\usepackage{mathpazo}
% begin custom commands
\newcommand{\Q}{\mathbb{Q}}
\newcommand{\R}{\mathbb{R}}
\newcommand{\N}{\mathbb{N}}
\newcommand{\F}{\mathcal{F}}
\newcommand{\X}{\mathcal{X}}
\newcommand{\W}{\mathcal{W}}
\newcommand{\Nor}{\mathcal{N}}
\newcommand{\U}{\mathcal{U}}
\newcommand{\Var}{\operatorname{Var}}
\newcommand{\E}{\operatorname{E}}
%\newcommand{\exp}{\operatorname{exp}}
\newcommand{\C}{\mathcal{C}}
\newcommand{\mc}{\mathcal}

% some shortcuts
\newcommand{\ds}{\displaystyle}
\newcommand{\arr}{\rightarrow}
\newcommand{\nop}[1]{}
\renewcommand{\hat}{\widehat}

% stuff for integrals
\newcommand{\intl}{\int\limits}
\newcommand{\rmd}{\mathrm{d}}


\newcommand{\norm}[1]{\left\lVert#1\right\rVert}
\usepackage[framemethod=latex]{mdframed}
\newenvironment{defn}[1]{\begin{mdframed}[backgroundcolor=blue!10,innertopmargin=15pt, nobreak=true,innerbottommargin=15pt]
		\textbf{#1 }
	}
	{ 
	\end{mdframed}
}

\newenvironment{important}{\begin{mdframed}[backgroundcolor=red!50,innertopmargin=15pt, innerbottommargin=15pt, nobreak=true]
		\Large
	}
	{ 
	\end{mdframed}
}

\newenvironment{lemma}{\begin{mdframed}[backgroundcolor=gray!50,innertopmargin=15pt, innerbottommargin=15pt, nobreak=true]
		\Large
	}
	{ 
	\end{mdframed}
}

\newenvironment{thm}[1]{\begin{mdframed}[nobreak=true,backgroundcolor=Emerald!10,innertopmargin=15pt, innerbottommargin=15pt]
		\textbf{#1 }
	}
	{ 
	\end{mdframed}
}



\newenvironment{trick}[1]{\begin{mdframed}[backgroundcolor=PineGreen!50,innertopmargin=15pt, innerbottommargin=15pt, nobreak=true]
		\textbf{#1 }
	}
	{ 
	\end{mdframed}
}


\usepackage{todonotes}
\newcommand{\TODO}[1]{\todo[inline]{\Large TODO:  #1}}





\DeclarePairedDelimiter\abs{\left|}{\right|}%



\setlength\abovedisplayskip{0pt}

\renewcommand{\familydefault}{\rmdefault}

% end custom commands

\begin{document}
	
	
	
	
	
	\maketitle
	
	
	
	\begin{multicols}{3}
		
		
		
		\textit{This summary provides a high level overview over most topics covered in the lecture Information Security taught at ETH Zürich by Prof .......}
		
		\part{Introduction to Cryptography}
		\section{Entropie}
		
	
		
		\begin{defn}{Entropie einer Zufallsvariable}
			Sei $X:\Omega \to \mathcal{X}$ mit Verteilung $P_X$ so dass $H(X) = H(P_X)$ so gilt
			
			\begin{align*}
				H(X) := - \sum_x P_X(x) \log P_X(x)
			\end{align*}
		\end{defn}
		
	
		
		\begin{thm}{Kettenregel}
			Für jedes Paar an Zufallsvariablen $X,Y$ gilt
			\begin{align*}
				H(X|Y) + H(Y) = H(X,Y) = H(Y|X) + H(X)
			\end{align*}
		\end{thm}
		
		\begin{thm}{Erweiterte Kettenregel}
			Es gilt durch mehrfaches anwenden der Kettenregel, dass
			
			\begin{align*}
				H(X_1,\ldots,X_n) &= H(X_1) + \sum_{i=2}^{n} H(X_i | X_1,\ldots X_{i-1})
			\end{align*}
			
			Ausgeschrieben wäre dies:
			$$H(X,Y,Z) = H(X) + H(Y|X) + H(Z|X,Y)$$
		\end{thm}
		
		
		\begin{defn}{Wechselseitige Information}
			Wir definieren 
			
			\begin{align*}
				I(X;Y) = H(X) - H(X|Y)
			\end{align*}
			
			als die Reduktion der Unsicherheit über $X$ durch Bekanntwerden von $Y$.
			
			Aufgrund der Kettenregel gilt
			
			\begin{align*}
				I(X;Y) &= H(X) + H(Y) - H(X,Y) = I(Y;X)
			\end{align*}
			
		\end{defn}
		
		
		\begin{thm}{Satz zur wechselseitigen Information}
			Für alle $X,Y$ gilt $I(X;Y)\geq 0$ mit Gleichheit wenn $X,Y$ unabhängig sind.	
		\end{thm}
		
		
		\begin{defn}{Kullback-Leibler-Divergenz}
			Die Kullback-Leibler-Divergenz zwischen zwei Zufallsvariablen $P,Q$ ist definiert durch
			
			\begin{align*}
				D_{KL} (P||Q) = -\sum_{\omega}P(\omega) \log\left( \frac{Q(\omega)}{P(\omega)} \right) = \E \left[\log\frac{P}{Q}\right]
			\end{align*}
		\end{defn}
		
		\begin{thm}{Gibbs Ungleichung}
			Es gilt $D_{KL}(P||Q)\geq 0 $ mit Gleichheit genau dann wenn $P=Q$.
		\end{thm}
		
		
		\begin{figure}
			\centering
			\ifx
			\includegraphics[width=1\linewidth]{overviewChpt1}
			\caption{Übersicht der relevanten Grössen}
			\label{fig:overviewchpt1} \fi
		\end{figure}
		\begin{defn}{Bedingte Gegenseitige Information}
			Die bedingte Gegenseitige Information, die $X$ über $Y$ gibt, gegeben $Z$ ist 
			
			\begin{align*}
				I(X;Y|Z) &= H(XZ) + H(YZ)- H(XYZ) - H(Z)
			\end{align*}
			
			Am besten erklärt man sich dies Intuitiv durch die Regel $I(X;Y) = H(X) + H(Y) - H(XY)$ wobei man immer auf $Z$ bedingt und dann auflöst.
			
		\end{defn}
	
		\begin{defn}
			This is a definition here\varepsilon \varepsilon \varepsilon \varepsilon \varepsilon \varepsilon \varepsilon \varepsilon \varepsilon \varepsilon \mu \mu \mu \lambda \rho \varpi \pi o \tau \upsilon 
		\end{defn}
		
		\newpage
		
	\end{multicols}
\end{document}
